\documentclass[11pt]{article}
\usepackage{geometry}
\usepackage{amsmath}
\usepackage{amssymb}
\usepackage{enumitem}
\usepackage{fancyhdr}
\usepackage{graphicx}
\usepackage{lastpage}
\usepackage{parskip}
\usepackage{siunitx}
\usepackage{listings} % to include code

\newif\ifclearpage

\newcommand{\name}{}
\newcommand{\class}{Math 155, Fall 2018}
\newcommand{\assignment}{Problem Set 4}
\newcommand{\duedate}{Due Thursday, October 4}
\newcommand{\rpm}{\sbox0{$1$}\sbox2{$\scriptstyle\pm$}
  \raise\dimexpr(\ht0-\ht2)/2\relax\box2 }

\clearpagetrue

\newcommand{\problem}[1]{\section*{#1}}
\newcommand{\solution}{\hrulefill}
\newcommand{\maybeclearpage}{\ifclearpage\clearpage\fi}

\renewcommand{\vec}[1]{\mathbf{#1}}

\geometry{margin=1in}

\fancypagestyle{primary}{
  \fancyhf{}
  \lhead{\name}
  \chead{\assignment}
  \rhead{\class}
  \lfoot{\duedate}
  \rfoot{\thepage{} of \pageref{LastPage}}}

\pagestyle{primary}

\setlist[enumerate]{label=(\alph*)}

\sisetup{per-mode=symbol}

\begin{document}

\problem{Chapter 3, Problem 11}
(Continuation of  Exercise 3.5) Return to the \textsc{wages} series.
\begin{enumerate}
	\item Consider the residuals from a least squares fit of a quadratic time trend.
	\item Perform a runs test on the standardized residuals and interpret the results.
	\item Calculate and interpret the sample autocorrelations for the standardized residuals.
	\item Investigate the normality of the standardized residuals (error terms). Consider histograms and normal probability plots. Interpret the plots. 
\end{enumerate}

\solution


\maybeclearpage
\problem{Chapter 3, Problem 13}
(Continuation of Exercise 3.7) Return to the \textsc{winnebago} time series.
\begin{enumerate}
	\item Calculate the least squares residuals from a seasonal-means plus linear time trend model on the logarithms of the sales time series.
	\item Perform a runs test on the standardized residuals and interpret the results.
	\item Calculate and interpret the sample autocorrelations for the standardized residuals.
	\item Investigate the normality of the standardized residuals (error terms). Consider histograms and normal probability plots. Interpret the plots. 
\end{enumerate}

\solution



\maybeclearpage
\problem{Chapter 3, Problem 16}
Suppose that a stationary time series $\{Y_t\}$, has an autocorrelation function of the form $\rho_k = \phi^k$ for $k > 0$, where $\phi$ is a constant in the range(-1,+1). 
\begin{enumerate}
	\item Show that $Var(\bar{y}) = \frac{\gamma_0}{n} \left[ \frac{1+\phi}{1-\phi} - \frac{(1-\phi^n}{(1-\phi)^2} \right] $ . (Hint: Use Equation (3.2.3) on page 28, the finit geometric sum $$\sum_{k=0}^n \phi^k = \frac{1-\phi^{1}}{1-\phi}, \text{, and the related sum } \sum_{k=0}^n k\phi^{k-1} = \frac{d}{d\phi} \left[ \sum_{k=0}^n \phi^k \right].$$
	\item If $n$ is large, argue that $Var(\bar{Y}) \approx \frac{\gamma_0}{n} \left[ \frac{1+\phi}{1-\phi} \right]$. 
	\item Plot $(1+\phi)/(1-\phi)$ for $\phi$ over the range -1 to +1. Interpret the plot in terms of the precision in estimating the process mean.
\end{enumerate}
\solution


\maybeclearpage
\problem{Chapter 4, Problem 1}
Use first principles to find the autocorrelation function for the stationary process defined by $$ Y_t = 5 + e_t - \frac{1}{2}e_{t-1} + \frac{1}{4}e_{t-2} $$

\solution


\maybeclearpage
\problem{Chapter 4, Problem 2}
Sketch the autocorrelation functions for the following MA(2) models with parameters as specified:
\begin{enumerate}
	\item $\theta_1 = 0.5$ and $\theta_2 = 0.4$.
	\item $\theta_1 = 1.2$ and $\theta_2 = -0.7$.
	\item $\theta_1 = -1$ and $\theta_2 = -0.6$.
\end{enumerate}

\solution

\maybeclearpage
\problem{Chapter 4, Problem 5}
Calculate and sketch the autocorrelation functions for each of the following AR(1) models. Plot for sufficient lags that the autocorrelation function has nearly died out.
\begin{enumerate}
	\item $\phi_1 = 0.6$.
	\item $\phi_1 = -0.6$.
	\item $\phi_1 = 0.95$. (Do out to 20 lags.)
	\item $\phi_1 = 0.3$.
\end{enumerate}

\solution

\maybeclearpage

\end{document}