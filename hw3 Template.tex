\documentclass[11pt]{article}
\usepackage{geometry}
\usepackage{amsmath}
\usepackage{amssymb}
\usepackage{enumitem}
\usepackage{fancyhdr}
\usepackage{graphicx}
\usepackage{lastpage}
\usepackage{parskip}
\usepackage{siunitx}
\usepackage{listings} % to include code

\newif\ifclearpage

\newcommand{\name}{}
\newcommand{\class}{Math 155, Fall 2018}
\newcommand{\assignment}{Problem Set 3}
\newcommand{\duedate}{Due Thursday, September 27}
\newcommand{\rpm}{\sbox0{$1$}\sbox2{$\scriptstyle\pm$}
  \raise\dimexpr(\ht0-\ht2)/2\relax\box2 }

\clearpagetrue

\newcommand{\problem}[1]{\section*{#1}}
\newcommand{\solution}{\hrulefill}
\newcommand{\maybeclearpage}{\ifclearpage\clearpage\fi}

\renewcommand{\vec}[1]{\mathbf{#1}}

\geometry{margin=1in}

\fancypagestyle{primary}{
  \fancyhf{}
  \lhead{\name}
  \chead{\assignment}
  \rhead{\class}
  \lfoot{\duedate}
  \rfoot{\thepage{} of \pageref{LastPage}}}

\pagestyle{primary}

\setlist[enumerate]{label=(\alph*)}

\sisetup{per-mode=symbol}

\begin{document}

\problem{Chapter 2, Problem 24}
Let $\{X_t\}$ be a time series in which we are interested. However, because the measurement process itself is not perfect, we actually observe $Y_t = X_t + e_t$. We assume that $\{X_t\}$ and $\{e_t\}$ are independent processes. We call $X_t$ the \textbf{signal} and $e_t$ the \textbf{measurement noise} or \textbf{error process}. \newline
If $\{X_t\}$ is stationary with autocorrelation function $\rho_k$, show that $\{Y_t\}$ is also stationary with $$Corr(Y_t,Y_{t=k}) = \frac{\rho_k}{1 + \sigma_e^2/\sigma_X^2} \text{ for } k \geq 1.$$
We call $\sigma_X^2 / \sigma_e^2$ the \textbf{signal-to-noise ratio}, or SNR. Note that the larger the SNR, the closer the autocorrelation function of the observed process $\{Y_t\}$ is to the autocorrelation function of the desired signal $\{X_t\}$. 

\solution


\maybeclearpage
\problem{Chapter 2, Problem 28}
(Random cosine wave extended) Suppose that $$Y_t = R\cos(2 \pi(ft + \phi)) \text{ for } t = 0, \rpm 1, \rpm 2, ...$$
where $0 < f < \frac{1}{2}$ is a fixed frequency and $R$ and $\phi$ are uncorrelated random variables and with $\phi$ uniformly distributed on the interval (0,1).
\begin{enumerate}
	\item Show that $E(Y_t) = 0$ for all $t$.
	\item Show that the process is stationary with $\gamma_k = \frac{1}{2}E(R^2)\cos(2 \pi fk)$. \newline Hint: Use the calculations leading up to Equation (2.3.4), on page 19.
\end{enumerate}

\solution



\maybeclearpage
\problem{Chapter 3, Problem 3}
Suppose $Y_t = \mu + e_t + e_{t-1}$. Find $Var(\bar{Y})$. Compare your answer to what would have been obtained if $Y_t = \mu + e_t$. Describe the effect that the autocorrelation in $\{Y_t\}$ has on $Var(\bar{Y})$.

\solution


\maybeclearpage
\problem{Chapter 3, Problem 5}
The data file \textsc{wages} contains monthly values of the average hourly wages (in dollars) for workers in the U.S. apparel and textile products industry for July 1981 through June 1987.
\begin{enumerate}
	\item Display and interpret the time series plot for these data.
	\item Use least squares to fit a linear time trend to this time series. Interpret the regression output.  Save the standardized residuals from the fit for further analysis. 
	\item Construct and interpret the time series plot of the standardized residuals from part (b). 
	\item Use least squares to fit a quadratic time trend to the wages time series. Interpret the regression output. Save the standardized residuals from the fit for further analysis.
	\item Construct and interpret the time series plot of the standardized residuals from part (d).
\end{enumerate}

\solution


\maybeclearpage
\problem{Chapter 3, Problem 7}
The data file \textsc{winnebago} contains monthly unit sales of recreational vehicles from Winnebago, Inc., from November 1966 through February 1972.
\begin{enumerate}
	\item Display and interpret the time series plot for these data.
	\item Use least squares to fit a line to these data. Plot the standardized residuals from the fit as a time series. Interpret the plot.
	\item Now take natural logarithms of the monthly sales figures and display and interpret the time series plot of the transformed values.
	\item Use least squares to fit a line to the logged data. Display and interpret the time series plot of the standardized residuals from this fit.
	\item Now use least squares to fit a seasonal-means plus linear time trend to the logged sales time series and save the standardized residuals for further analysis. Check the statistical significance of each of the regression coefficients in the model.
	\item Display the time series plot of the standardized residuals obtained in  part (e). Interpret the plot.
\end{enumerate}

\solution

\maybeclearpage


\end{document}