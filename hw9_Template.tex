\documentclass[11pt]{article}
\usepackage{geometry}
\usepackage{amsmath}
\usepackage{amssymb}
\usepackage{enumitem}
\usepackage{fancyhdr}
\usepackage{graphicx}
\usepackage{lastpage}
\usepackage{parskip}
\usepackage{siunitx}
\usepackage{listings} % to include code

\newif\ifclearpage

\newcommand{\name}{}
\newcommand{\class}{Math 155, Fall 2018}
\newcommand{\assignment}{Problem Set 9}
\newcommand{\duedate}{Due Thursday, December 6}
\newcommand{\rpm}{\sbox0{$1$}\sbox2{$\scriptstyle\pm$}
  \raise\dimexpr(\ht0-\ht2)/2\relax\box2 }

\clearpagetrue

\newcommand{\problem}[1]{\section*{#1}}
\newcommand{\solution}{\hrulefill}
\newcommand{\maybeclearpage}{\ifclearpage\clearpage\fi}

\renewcommand{\vec}[1]{\mathbf{#1}}

\geometry{margin=1in}

\fancypagestyle{primary}{
  \fancyhf{}
  \lhead{\name}
  \chead{\assignment}
  \rhead{\class}
  \lfoot{\duedate}
  \rfoot{\thepage{} of \pageref{LastPage}}}

\pagestyle{primary}

\setlist[enumerate]{label=(\alph*)}

\sisetup{per-mode=symbol}

\begin{document}

\problem{Chapter 7, Problem 17}
Simulate an ARMA(1,1) series with $\phi = 0.7$, $\theta = 0.4$, and $n = 72$.
\begin{enumerate}
\item Find the method-of-moments estimates of $\phi$ and $\theta$
\item Find the conditional least squares estimates of $\phi$ and $\theta$ and compare them with part (a).
\item Find the maximum likelihood estimates of $\phi$ and $\theta$ and compare them with parts (a) and (b).
\item Repeat parts (a), (b), and (c) with a new simulated series using the same parameters and same sample size. Compare your new results with your results from the first simulation.
\end{enumerate}

\solution


\maybeclearpage
\problem{Chapter 7, Problem 29}
The data file named \textsc{robot} contains a time series obtained from an industrial robot. The robot was put through a sequence of maneuvers, and the distance from a desired ending point was recorded in inches. This was repeated 324 times to form the time series. 
\begin{enumerate}
\item Estimate the parameters of an AR(1) model for these data. 
\item Estimate the parameters of an IMA(1,1) model for these data. 
\item What are the theoretical partial autocorrelations for this model?
\item Compare the results from parts (a) and (b) in  terms of AIC.
\end{enumerate}

\solution


\maybeclearpage
\problem{Chapter 7, Problem 30}
The data file named \textsc{days} contains accounting data from the Winegard Co. of Burlington, Iowa. The data are the number of days until Winegard receives payment for 130 consecutive orders from a particular distributor of Winegard products. (The name of the distributor must remain anonymous for confidentiality reasons.)
\begin{enumerate}
\item Display the time series plot of the data. Based on this information, do these data appear to come from a stationary or nonstationary process?
\item Replace each of the unusual values with a value of 35 days, a much more typical value, and then estimate the parameters of an MA(2) model.
\item Now assume an MA(5) model and estimate the parameters. Compare these results with those obtained in part(a). 
\end{enumerate}

\solution

\maybeclearpage
\problem{Chapter 8, Problem 7}
Fit an AR(3) model by maximum likelihood to the square root of the hare abundance series (filename \textsc{hare}).
\begin{enumerate}
\item Plot the sample ACF of the residuals. Comment on the size of the correleations.
\item Calculate the Ljang-Box statistic summing to $K=9$. Does this statistic support the AR(3) specification?
\item Perform a runs test on the residuals and comment on the results.
\item Display the quantile-quantile normal plot of the residuals. Comment on the plot.
\item Perform the Shapiro-Wilk test of normality on the residuals.
\end{enumerate}

\solution


\maybeclearpage
\problem{Chapter 8, Problem 9}
The data file named \textsc{robot} contains a time series obtained from an industrial robot. The robot was put through a series of maneuvers, and the distance from a desired ending points was recorded in inches. This was repeated 324 times to form the time series. Compare the fits of an AR(1) model and an IMA(1,1) model for these data in terms of the diagnostic tests discussed in this chapter.

\solution


\maybeclearpage
\problem{Chapter 8, Problem 10}
The data file names \textsc{deere3} contains 57 consecutive values from a complex machine tool at Deere \& Co. The values given are deviations from a target value in units of ten millionths of an inch. The process employs a control mechanism that resets some of the parameters of hte machine tool depending on the magnitude of deviation from target of the last item produced. Diagnose the fit of an AR(1) model for these data in terms of the tests discussed in this chapter. 

\solution

\maybeclearpage

\end{document}